\section{Market Analysis}

\subsection{Industry Analysis}
\subsubsection{Size and Growth Rate}
% TODO: Cần nghiên cứu và bổ sung dữ liệu về quy mô thị trường và tốc độ tăng trưởng
\{TBA - Cần cung cấp dữ liệu về giá trị thị trường và tốc độ tăng trưởng dự kiến\}

\subsubsection{Key Trends}
% TODO: Cần phân tích các xu hướng chính
\textbf{Technological Trends:}
\begin{itemize}
    \item AI Revolution và ứng dụng trong giáo dục
    \item Xu hướng học tập số hóa (digital learning)
    \item \{TBA - Cần bổ sung thêm xu hướng công nghệ\}
\end{itemize}

\textbf{Social and Economic Trends:}
\begin{itemize}
    \item \{TBA - Xu hướng xã hội và kinh tế tác động đến ngành\}
\end{itemize}

\subsection{Competitive Analysis (Porter's 5 Forces)}
\subsubsection{Industry Rivalry}
% Nội dung được di chuyển từ chương Competition Analysis
\{TBA - Cần phân tích chi tiết các đối thủ cạnh tranh chính\}

\textbf{Main Competitors:}
\begin{itemize}
    \item PhotoMath - Ứng dụng giải toán bằng camera
    \item WolframAlpha - Công cụ tính toán và giải bài tập
    \item Khan Academy - Nền tảng học tập trực tuyến
    \item \{TBA - Cần bổ sung thêm đối thủ cạnh tranh\}
\end{itemize}

\subsubsection{Threat of New Entrants}
% TODO: Phân tích rào cản gia nhập
\{TBA - Rào cản gia nhập cao hay thấp?\}

\subsubsection{Bargaining Power of Suppliers}
% TODO: Phân tích sức mạnh thương lượng của nhà cung cấp
\{TBA - Mức độ ảnh hưởng của các nhà cung cấp\}

\subsubsection{Bargaining Power of Customers}
% TODO: Phân tích sức mạnh thương lượng của khách hàng
\{TBA - Khách hàng có nhiều lựa chọn không?\}

\subsubsection{Threat of Substitute Products}
% TODO: Phân tích các sản phẩm thay thế
\{TBA - Các giải pháp khác có thể đáp ứng nhu cầu khách hàng\}

\subsection{Target Market}
\subsubsection{Market Segmentation}
% TODO: Phân chia thị trường dựa trên các tiêu chí
\textbf{Geographic Segmentation:}
\begin{itemize}
    \item Primary: Vietnam
    \item \{TBA - Kế hoạch mở rộng địa lý\}
\end{itemize}

\textbf{Demographic Segmentation:}
\begin{itemize}
    \item Age: 16-24 years old
    \item Education level: Grade 10 through university
    \item \{TBA - Cần bổ sung thêm tiêu chí nhân khẩu học\}
\end{itemize}

\textbf{Psychographic and Behavioral Segmentation:}
\begin{itemize}
    \item \{TBA - Phân tích tâm lý và hành vi khách hàng\}
\end{itemize}

\subsubsection{Target Market Selection}
% Nội dung được di chuyển từ chương Feasibility
\textbf{Primary Target:} Students in Vietnam from grade 10 through the university level (ages 16-24).

\textbf{Rationale:} \{TBA - Cần giải thích lý do lựa chọn phân khúc này\}

\subsubsection{Customer Persona}
% TODO: Tạo profile chi tiết của khách hàng lý tưởng
\{TBA - Cần tạo customer persona chi tiết\}

\subsection{Product Desirability}
% Nội dung được di chuyển từ chương Feasibility
\subsubsection{Desirability}
To test desirability, we developed the following concept statement to solicit feedback.

\medskip

{
    \setlength{\fboxsep}{10pt}
    \noindent\fbox{%
        \parbox{\textwidth}{%
            \textbf{Concept Statement}: AI-Powered STEM Tutor

            \textit{Product Description}: A mobile application that functions as an AI-powered tutor for STEM subjects. The
            app uses a novel two-step AI architecture:
            \begin{itemize}
            \item A "Solver Agent" generates a correct, hidden program to solve a problem, ensuring accuracy.
            \item A "Tutor Agent" then uses the program's logic to guide the student through a multi-tiered Socratic
            learning process, from providing hints to offering a full step-by-step guide.
            \end{itemize} A key feature of the proposed product prototype is a "self-documenting" engine, where the AI's
            reasoning is embedded as comments within the solver program, allowing the Tutor Agent to provide more
            effective, context-aware guidance.

            \textit{Target Market}: Students in Vietnam from grade 10 through the university level (ages 16-24).

            \textit{Benefits of the Product}: The app addresses a key problem with existing tools like PhotoMath, which
            often provide answers without fostering genuine learning. By focusing on the process of problem-solving, the
            app helps students develop critical thinking and build lasting competence, aligning with modern educational
            philosophies.  

            \textit{Positioning}: The app will be positioned as a true "learning aid" that teaches students how to solve
            problems, differentiating it from "answer engines" that can encourage academic dishonesty.   
        }
    }
}

\subsubsection{Demand}
\{TBA - Cần nghiên cứu và phân tích nhu cầu thị trường\}

\subsection{SWOT Analysis}
\subsubsection{Strengths}
% TODO: Phân tích các yếu tố tích cực nội bộ
\begin{itemize}
    \item \{TBA - Điểm mạnh nội bộ\}
\end{itemize}

\subsubsection{Weaknesses}
% TODO: Phân tích các yếu tố tiêu cực nội bộ
\begin{itemize}
    \item \{TBA - Điểm yếu nội bộ\}
\end{itemize}

\subsubsection{Opportunities}
% TODO: Phân tích các yếu tố bên ngoài có thể tận dụng
\begin{itemize}
    \item \{TBA - Cơ hội từ môi trường bên ngoài\}
\end{itemize}

\subsubsection{Threats}
% TODO: Phân tích các yếu tố bên ngoài có thể gây hại
\begin{itemize}
    \item \{TBA - Mối đe dọa từ môi trường bên ngoài\}
\end{itemize}