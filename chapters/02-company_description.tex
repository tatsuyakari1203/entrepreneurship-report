\def\code#1{\texttt{#1}}

\section{Company Description}

\subsection{History and Development}
% Nội dung được di chuyển và tái cấu trúc từ chương Ideation
We are in the middle of a paradigm shift of how we interact with computers, language, and information. The current so-called "AI Revolution" can have its origin traced back to late 2016-early 2017 when "AI" (as we refer today) was able to synthesise new data as deepfakes, no longer limited to mere classification. As deepfake matures, the world started to ponder: If deepfakes are possible, can we push this generative into other domains? After a brief period of silent refinement and research came the answer: A resounding yes. GPT-1 debutted in 2018, followed by GPT-2 in 2019; finally culminated into the start of the "AI Revolution", GPT-3, colloquially known as "ChatGPT".

The world took notice very quickly. GPT-3 and later models excelled at natural language processing (NLP), generating responses the likes of Cleverbot could never hoped to match. But, a sword to protect is also a sword for violence, as AI-based cheating became more and more prolific amongst student. Why bother labouring away at assignments when you could ask to "please generate an essay on \$\{a topic\}"? An arms race have begun: Students who wanted a quick way out, and educators who wanted to curb the rampant low-effort submissions. Pandora's Box had been opened, there was no going back. There was only more education on how to wield AI as a tool, which got us thinking.

It was in this context that CogniMind was founded in 2024 by a team of four passionate technology and business students who shared a common concern about current learning methodologies. We recognized that existing learning support platforms typically only provide final answers, inadvertently creating dependency and reducing students' independent thinking capabilities.

The idea for CogniMind emerged from our desire to change this paradigm. Instead of just providing "what" (the answer), we focus on "why" and "how" (the thinking process). The project began as an assignment in our Entrepreneurship course and quickly evolved into a genuine mission: building an AI learning companion that helps students not only solve problems but truly understand and master knowledge. This challenge got us thinking about how to harness AI as a proper educational tool rather than a shortcut.

\subsection{Mission, Vision, and Core Values}

\textbf{Mission Statement:}
"Empowering learners by transforming the search for solutions into a journey of discovery and critical thinking. We use AI to guide, not replace, helping students build solid knowledge foundations and self-problem-solving skills."

\textbf{Vision Statement:}
"To become the leading AI learning companion in Southeast Asia, reshaping the future of personalized education and nurturing a generation of independent, creative learners."

\textbf{Core Values:}
\begin{itemize}
    \item \textbf{Learner-Centric:} All our decisions stem from the goal of enhancing students' long-term learning experience and outcomes.
    \item \textbf{Constant Innovation:} We are committed to pioneering the application of advanced AI technology to create breakthrough and effective educational solutions.
    \item \textbf{Academic Integrity:} We encourage honesty in learning; our tools are designed to support thinking, not to enable cheating.
    \item \textbf{Accessibility:} We strive to ensure that every student, regardless of circumstances, can access high-quality learning support.
    \item \textbf{Excellence:} We set the highest standards for our technology, educational content, and support services.
\end{itemize}

\subsection{Product and Service}
LLMs' strength lies in NLP, i.e., its ability to crunch through hundreds of pages of text, teasing out details, and
based on its training with sufficient guidance (i.e., prompting), could help a well-trained individual to cut down on
time needed to meaningfully parse through a document. Examples of this can be seen in the existence of numerous AI
academic research aids: The LLM with their tools will blast through a research question, finding the relevant papers,
distill them down to answer the question, and leave it for the human to judge whether the material found is suitable
for their current objective. It is this abililty to aid humans that we wanted to harness.

Key words: "well-trained individual". What about the not well-trained? We have seen in the past tools such as PhotoMath,
WolframAlpha being used to solve a problem for a student to copy down and call it done. As far as the education system
is concerned, you have proved your ability to solve a problem, but as far as your brain is concerned, you have not
learnt anything. And as digital note-taking become more or more prevalent in K-12, not just as college/university
students, we wanted to let AI empower K-12 and university students of STEM fields, in much the same way AI is
empowering academia.

We would like to propose: "\textbf{CogniMind}, a notebook note-taking app, with AI features tailored towards students".
We want to achieve this by purposefully restricting the LLM's capability into 3 stages: Priming the user with hints on
required knowledge; Socratic questioning to guide the user through; and full solution. The goal is for the user to
connect the dots themselves, rather than spoonfeeding.

\subsection{Unique Selling Proposition (USP)}
Our startup will \textbf{strictly focus on STEM subjects}. The law of physics is the same regardless of countries,
whereas culture can differ wildly from one region to the next, sometimes within the same country. An LLM would need a
full-scale anthropology research project culminating into thousands upon thousands of pages of text describing in
painstaking detail every nuances; the scale of which is a state-sponsored project. As LLMs can hallucinate (this is by
nature as a token (i.e., morpheme, loosely) predictor), by focusing strictly on STEM subjects where there is a
"universal truth", we can prevent misinformation on factual subjects that can distort a student's view, especially on
sensitive topics.

\textbf{Key Differentiators:}
\begin{itemize}
    \item Three-tiered learning approach: Hints → Socratic questioning → Full solution
    \item Focus on learning process rather than providing direct answers
    \item STEM-specific to ensure accuracy and avoid cultural biases
    \item AI-powered note-taking integration
\end{itemize}

\subsection{Legal Status and Ownership}

\textbf{Business Entity Type:}
CogniMind is planned to be established as a Limited Liability Company (LLC) in Vietnam. This legal structure is flexible and suitable for an early-stage startup, helping protect the personal assets of founders and simplifying initial procedures.

\textbf{Ownership Structure:}
Founder equity is distributed based on each member's primary role, responsibilities, and expected contribution in the initial phase. The proposed structure is as follows:
\begin{itemize}
    \item \textbf{Hao (Lead - Product \& Strategy):} 35\%
    \item \textbf{Hieu (EduTech \& Data):} 25\%
    \item \textbf{Quoc Duy (Development \& AI):} 25\%
    \item \textbf{Vinh An (Operations \& QA):} 15\%
\end{itemize}

Additionally, the company will allocate \textbf{10\% equity for an Employee Stock Option Pool (ESOP)}. This pool will be used to attract and retain key talent in the future, ensuring sustainable team development.

\textbf{Licenses and Compliance:}
The company will register for business operations with primary activities in information technology and software development. We are committed to strict compliance with Vietnamese law, particularly Decree 13/2023/ND-CP on Personal Data Protection, to ensure user information security.

\subsection{Timeline and Intellectual Property}

\textbf{Development Milestones:}
\begin{itemize}
    \item \textbf{Q3, 2024:} CogniMind concept formed and developed in Entrepreneurship course
    \item \textbf{Q4, 2024:} Founding team established, business model refined, and internal prototype developed
    \item \textbf{Q1, 2025:} Detailed business plan completed with positive initial feedback from advisors
    \item \textbf{Target - Q3, 2025:} Beta version launch for limited user group to collect feedback
    \item \textbf{Target - Q1, 2026:} Official application launch in Vietnamese market
\end{itemize}

\textbf{Intellectual Property:}
\begin{itemize}
    \item \textbf{Trademark:} The "CogniMind" name and logo are in preparation for trademark registration in Vietnam
    \item \textbf{Trade Secrets:} Our core intellectual property is the proprietary RAG (Retrieval-Augmented Generation) system architecture and AI model fine-tuning methods for directed learning purposes. These elements will be protected as trade secrets.
\end{itemize}