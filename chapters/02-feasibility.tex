\section{Feasibility}

\subsection{Product desiribility}
\subsubsection{Desiribility}
To test desiribility, we developed the following concept statement to solicit feedback.

\medskip

{
    \setlength{\fboxsep}{10pt}
    \noindent\fbox{%
        \parbox{\textwidth}{%
            \textbf{Concept Statement}: AI-Powered STEM Tutor

            \textit{Product Description}: A mobile application that functions as an AI-powered tutor for STEM subjects. The
            app uses a novel two-step AI architecture:
            \begin{itemize}
            \item A "Solver Agent" generates a correct, hidden program to solve a problem, ensuring accuracy.
            \item A "Tutor Agent" then uses the program's logic to guide the student through a multi-tiered Socratic
            learning process, from providing hints to offering a full step-by-step guide.
            \end{itemize} A key feature of the proposed product prototype is a "self-documenting" engine, where the AI's
            reasoning is embedded as comments within the solver program, allowing the Tutor Agent to provide more
            effective, context-aware guidance.

            \textit{Target Market}: Students in Vietnam from grade 10 through the university level (ages 16-24).

            \textit{Benefits of the Product}: The app addresses a key problem with existing tools like PhotoMath, which
            often provide answers without fostering genuine learning. By focusing on the process of problem-solving, the
            app helps students develop critical thinking and build lasting competence, aligning with modern educational
            philosophies.  

            \textit{Positioning}: The app will be positioned as a true "learning aid" that teaches students how to solve
            problems, differentiating it from "answer engines" that can encourage academic dishonesty.   
        }
    }
}
\subsubsection{Demand}
\{TBA\}

\subsection{Industry attractiveness}
\{TBA\}

\subsection{Target market feasibility}
\{TBA\}

\subsection{Resource}
\subsubsection{GUI}
Notetaking app is an idea that has been done to death, as such, it is highly feasible. To cut down on
 iteration time, we opt to use React Native (\url{https://reactnative.dev}) to be our Graphical User Interface
 (GUI) framework. Based on the well-known web UI framework React with minor changes, coupled with extensive open-source
 component library, we can cut down on time-to-MVP while simultaneously provide an interface in the browser or
 Electron-based web apps without suffering from parallel technology stacks.

\subsubsection{LLMs}
The core technology behind the AI half of our notetaking app are the LLMs. We choose to roll with
 Google's Gemini with possibility to expand to other providers such at OpenAI's GPT-* models. Both companies provide
 API to interact with their LLMs under pay-as-you-go pricing model allowing for flexibility in cost. Furthermore,
 having access to APIs allows us to tweak the system prompt (i.e., personality, tone of speech, etc.) according to our
 need rather than being confined to the defaults. In addition, LLMs are no longer confined to text, they can
 participate in agentic workflow, which we need to tailor the responses according to our and our customers' needs.

\subsubsection{Input text}
Aiming for K-12 students, a thorny challenge is the acquisition of textbooks for refining
 which is copyright. While the concepts being introduced in the textbooks are not copyrightable, the ways the concepts
 are introduced are under the so-called Idea-Expression Dichotomy. To avoid copyright issues, we opt to use a method
 similar to a clean room implementation: One person will be reading through standard issue MoET textbooks, noting down
 the concepts being introduced and systemise them into a "skeleton"; another person will use reference other material
 to repopulate and refine the skeleton into a so-called "graph of knowledge". Using this method, we can leverage the
 existing "knowledge" embedded in the LLM itself while adding guard rails to prevent leakage of "higher knowledge",
 i.e., \textbf{not} introducing trigonometry in guiding a user who has only learnt Pythagorean Theorem, etc. 

 In later revisions, this can be expanded to include actual sample problems and solutions curated and created in-house
 to further nudge the LLM into following the expected baseline method, with human-in-the-loop refinement through
 collaboration with human field experts (teachers, lecturers, etc.) to further enhance the LLM's response.
