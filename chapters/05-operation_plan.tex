\section{Operation Plan}

\subsection{Operational Workflow}

CogniMind's operational plan is designed to ensure flexibility, efficiency, and scalability by applying modern methodologies in product development and service management. We adopt an Agile model combined with DevOps to optimize the product and service lifecycle.

\subsubsection{Product Development Lifecycle}

Our product development follows a structured approach with continuous feedback integration:

\begin{enumerate}
    \item \textbf{Requirement Gathering:} Continuously collect user feedback through surveys, interviews, and behavioral data analysis. The strategy team researches market trends and educational developments to identify priority features.
    
    \item \textbf{Design \& Prototyping:} The UX/UI team creates wireframes and interactive prototypes. Simultaneously, the AI team designs and tests prompt structures and workflows for Large Language Models (LLMs).
    
    \item \textbf{Development \& QA:} Products are developed in short cycles (2-week sprints). After each sprint, the QA team conducts comprehensive testing, from unit tests and integration tests to user acceptance testing (UAT) to ensure quality.
    
    \item \textbf{Deployment \& Monitoring:} We use CI/CD (Continuous Integration/Continuous Deployment) processes to automate new version deployments. Performance and error monitoring systems operate 24/7 to detect and resolve issues promptly.
\end{enumerate}

\subsubsection{Service Delivery Process}

Our service delivery is optimized for user experience and educational effectiveness:

\begin{enumerate}
    \item \textbf{User Onboarding:} The registration process is streamlined. New users receive a brief tour of key features to start using the platform immediately.
    
    \item \textbf{Core Service (AI Tutoring):} When users submit requests, the system processes them through a complex AI workflow to generate pedagogically sound suggestions and explanations, ensuring accuracy and appropriateness for the learner's level.
    
    \item \textbf{Tracking \& Analytics:} The system collects anonymized data about the learning process, helping users track their progress and helping CogniMind better understand user behavior for product improvement.
\end{enumerate}

\subsubsection{Customer Support Process}

We maintain high-quality customer support through multiple channels:

\begin{enumerate}
    \item \textbf{Reception:} Users can submit support requests via email, integrated chatbot, and social media channels.
    
    \item \textbf{Classification \& Processing:} Requests are automatically categorized (technical issues, payment questions, feature suggestions) and routed to appropriate departments. Our support team commits to responding within 24 hours.
    
    \item \textbf{Resolution \& Improvement:} After resolving issues, feedback and bug reports are compiled and added to the product development backlog for improvement in future versions.
\end{enumerate}

\subsection{Technology and Equipment}

CogniMind's technology platform is built on modern frameworks with high scalability and cost optimization capabilities.

\subsubsection{GUI Framework}
\textbf{React Native:} Chosen as the primary framework for application development. The main advantage is the ability to build applications for both iOS and Android from a single codebase, significantly saving development time and costs. The rich component ecosystem and strong community support are also major benefits.

\subsubsection{LLMs Infrastructure}
\begin{itemize}
    \item \textbf{Core Engine:} Using \textbf{Google Gemini API} (e.g., Gemini 1.5 Pro) as the main AI engine due to its multimodal processing capabilities, large context window, and optimal performance.
    \item \textbf{Backup \& Alternative:} \textbf{OpenAI GPT-4 Series} will be used as a backup model and for comparison, ensuring service continuity.
    \item \textbf{Architecture:} We build a proprietary \textbf{Agentic Workflow} that allows AI to automatically perform a series of reasoning steps, retrieve knowledge from RAG database, and self-verify answers before providing suggestions to users.
\end{itemize}

\subsubsection{Content \& Data Development}
\textbf{Clean Room Implementation:} To avoid any copyright issues, we do not directly copy content from textbooks. Instead, we analyze the curriculum structure of the Ministry of Education and Training to build a "knowledge skeleton."

\textbf{Knowledge Graph:} From this skeleton, our expert team will independently build examples, exercise types, and detailed solutions. This data is structured into a knowledge graph, helping AI efficiently retrieve and link information.

\textbf{Human-in-the-Loop:} All AI-generated content will be reviewed and refined by our team of teachers and subject experts to ensure accuracy, appropriateness, and pedagogical value.

\subsubsection{Additional Technology Requirements}
\textbf{Development Tools:}
\begin{itemize}
    \item \textbf{IDE:} Visual Studio Code with React Native extensions for cross-platform development
    \item \textbf{Version Control:} Git with GitHub for collaborative development and code management
    \item \textbf{CI/CD Pipeline:} GitHub Actions for automated testing, building, and deployment
    \item \textbf{Testing Frameworks:} Jest for unit testing, Detox for end-to-end testing
\end{itemize}

\textbf{Infrastructure:}
\begin{itemize}
    \item \textbf{Cloud Hosting:} Google Cloud Platform (GCP) for seamless integration with Gemini API and auto-scaling capabilities
    \item \textbf{Database Systems:} PostgreSQL for structured data (user profiles, payments) and Pinecone for vector database (RAG system)
    \item \textbf{CDN:} Cloudflare for global content delivery and performance optimization
    \item \textbf{Monitoring:} Google Cloud Monitoring and Analytics for system performance tracking
\end{itemize}

\textbf{Security:}
\begin{itemize}
    \item \textbf{Data Encryption:} AES-256 encryption for data at rest and TLS 1.3 for data in transit
    \item \textbf{Authentication:} OAuth 2.0 and JWT tokens for secure user authentication
    \item \textbf{Privacy Compliance:} GDPR and local privacy regulation compliance tools and procedures
\end{itemize}

\subsection{Location}

\subsubsection{Main Office}
\textbf{Primary Location:} Located in \textbf{Ho Chi Minh City}, potentially in a modern co-working space such as WeWork or Dreamplex in District 1, District 2, or near the High-Tech Park (Thu Duc City) for easy access to high-quality technology talent from universities.

\subsubsection{Work Model}
\textbf{Hybrid Approach:} We adopt a flexible \textbf{Hybrid work model (3 days in office, 2 days remote)} to attract talent nationwide, increase employee satisfaction, and optimize operational costs.

\textbf{Benefits of Our Location Strategy:}
\begin{itemize}
    \item Access to top-tier technology talent from leading universities
    \item Proximity to the startup ecosystem and potential partners
    \item Cost-effective compared to traditional office leasing
    \item Flexibility to scale up or down based on team growth
    \item Enhanced work-life balance for team members
\end{itemize}

\subsubsection{Remote Work Infrastructure}
\begin{itemize}
    \item Cloud-based development and collaboration tools
    \item Secure VPN access for remote team members
    \item Regular virtual meetings and team building activities
    \item Flexible working hours to accommodate different time zones
\end{itemize}

\subsection{Supply Chain and Partner Management}

\subsubsection{Key Suppliers}
\textbf{Technology Providers:}
\begin{itemize}
    \item \textbf{Google:} Gemini API and Google Cloud Platform (GCP) services
    \item \textbf{OpenAI:} GPT-4 series as backup and comparison models
    \item \textbf{AWS:} Alternative cloud infrastructure provider
\end{itemize}

\textbf{Content Development:}
\begin{itemize}
    \item \textbf{Educational Experts:} Teachers, lecturers, and education specialists working as freelance collaborators to build and review educational content
    \item \textbf{Subject Matter Experts:} Specialists in mathematics, physics, chemistry, and other subjects for content validation
\end{itemize}

\subsubsection{Strategic Partners}
\textbf{Educational Institutions:}
\begin{itemize}
    \item \textbf{Pioneer Schools:} Partnerships with technology-forward high schools and universities (e.g., FPT Schools, Vinschool, RMIT University, Fulbright University) for pilot programs
    \item \textbf{Teacher Training Centers:} Collaboration for professional development and feedback collection
\end{itemize}

\textbf{Technology Partners:}
\begin{itemize}
    \item \textbf{Non-competing EdTech Companies:} Cross-integration with Learning Management Systems (LMS) or language learning applications
    \item \textbf{Educational Technology Distributors:} Partners for market expansion and institutional sales
\end{itemize}

\subsubsection{Partner Management}
\textbf{Service Level Agreements (SLAs):}
\begin{itemize}
    \item Clear SLAs with technology providers ensuring 99.9\% uptime
    \item Response time commitments for technical support
    \item Data security and privacy compliance requirements
\end{itemize}

\textbf{Partnership Evaluation:}
\begin{itemize}
    \item Quarterly performance reviews with educational partners
    \item Regular feedback collection and improvement planning
    \item Annual strategic partnership assessments
    \item Continuous monitoring of partner satisfaction and mutual value creation
\end{itemize}